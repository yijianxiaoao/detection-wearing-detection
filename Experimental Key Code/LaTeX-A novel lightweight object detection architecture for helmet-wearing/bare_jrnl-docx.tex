\documentclass[journal]{IEEEtran}

\ifCLASSINFOpdf
\else
   \usepackage[dvips]{graphicx}
\fi
\usepackage{url}
\hyphenation{op-tical net-works semi-conduc-tor}
\usepackage{graphicx}


\begin{document}

\title{Preparation of Papers for IEEE Signal Processing Letters (5-page limit)}

\author{First A. Author, \IEEEmembership{Fellow, IEEE}, Second B. Author, and Third C. Author, Jr., \IEEEmembership{Member, IEEE}
\thanks{This paragraph of the first footnote will contain the date on which you submitted your paper for review. It will also contain support information, including sponsor and financial support acknowledgment. For example, ``This work was supported in part by the U.S. Department of Commerce under Grant BS123456.'' }
\thanks{The next few paragraphs should contain the authors' current affiliations, including current address and e-mail. For example, F. A. Author is with the National Institute of Standards and Technology, Boulder, CO 80305 USA (e-mail: author@boulder.nist.gov).}
\thanks{S. B. Author, Jr., was with Rice University, Houston, TX 77005 USA. He is now with the Department of Physics, Colorado State University, Fort Collins, CO 80523 USA (e-mail: author@lamar.colostate.edu).}}

\markboth{Journal of \LaTeX\ Class Files, Vol. 14, No. 8, August 2015}
{Shell \MakeLowercase{\textit{et al.}}: Bare Demo of IEEEtran.cls for IEEE Journals}
\maketitle

\begin{abstract}
These instructions give you guidelines for preparing papers for IEEE Signal Processing Letters. Use this document as a template if you are using Microsoft Word 6.0 or later. Otherwise, use this document as an instruction set. The electronic file of your paper will be formatted further at IEEE. Paper titles should be written in uppercase and lowercase letters, not all uppercase. Do not write ``(Invited)'' in the title. Full names of authors are preferred in the author field, but are not required. Put a space between authors’ initials. The abstract must be a concise yet comprehensive reflection of what is in your article. In particular, the abstract must be self-contained, without abbreviations, footnotes, or references. It should be a microcosm of the full article. The abstract is typically between 100--175 words. The abstract must be written as one paragraph, and should not contain displayed mathematical equations or tabular material. The abstract should include three or four different keywords or phrases, as this will help readers to find it. It is important to avoid over-repetition of such phrases as this can result in a page being rejected by search engines. Ensure that your abstract reads well and is grammatically correct.
\end{abstract}

\begin{IEEEkeywords}
Enter key words or phrases in alphabetical order, separated by commas. For a list of suggested keywords, send a blank e-mail to keywords@ieee.org or visit \url{http://www.ieee.org/organizations/pubs/ani_prod/keywrd98.txt}
\end{IEEEkeywords}


\IEEEpeerreviewmaketitle



\section{Introduction}

\IEEEPARstart{T}{his} document is a template for Microsoft Word versions 6.0 or later. If you are reading a paper or PDF version of this document, please download the electronic file, trans\_jour.docx, from the IEEE Web site at \url{www.ieee.org/authortools} so you can use it to prepare your manuscript. If you would prefer to use LaTeX, download IEEE's LaTeX style and sample files from the same Web page. You can also explore using the Overleaf editor at {https://www.overleaf.com/blog/278-how-to-use-overleaf-with-ieee-collabratec-your-quick-guide-to-getting-started\#.Vp6tpPkrKM9}

If your paper is intended for a conference, please contact your conference editor concerning acceptable word processor formats for your particular conference.  


\section{Guidelines For Manuscript Preparation}

When you open trans\_jour.docx, select ``Page Layout'' from the ``View'' menu in the menu bar (View Page Layout), (these instructions assume MS 6.0. Some versions may have alternate ways to access the same functionalities noted here). Then, type over sections of trans\_jour.docx or cut and paste from another document and use markup styles. The pull-down style menu is at the left of the Formatting Toolbar at the top of your {\em Word} window (for example, the style at this point in the document is ``Text''). Highlight a section that you want to designate with a certain style, and then select the appropriate name on the style menu. The style will adjust your fonts and line spacing. Do not change the font sizes or line spacing to squeeze more text into a limited number of pages. Use italics for emphasis; do not underline. 

To insert images in {\em Word}, position the cursor at the insertion point and either use Insert $\mid$ Picture $\mid$ From File or copy the image to the Windows clipboard and then Edit $\mid$ Paste Special $\mid$ Picture (with ``float over text'' unchecked). 

IEEE will do the final formatting of your paper. If your paper is intended for a conference, please observe the conference page limits. 


\subsection{Information for Authors}

{\em IEEE Signal Processing Letters} allows only four-page articles. A fifth page is allowed for ``References'' only, though ``References'' may begin before the fifth page. Author biographies or photographs are not allowed in Signal Processing Letters. Please review the Information for Authors at for {\em IEEE Signal Processing Letters:} https://signalprocessingsociety.org/publications-resources/ieee-signal-processing-letters/information-authors-spl

\begin{figure}
\centerline{\includegraphics[width=\columnwidth]{fig1.png}}
\caption{Magnetization as a function of applied field. Note that ``Fig.'' is abbreviated. There is a period after the figure number, followed by two spaces. It is good practice to explain the significance of the figure in the caption.}
\end{figure}

\begin{table}
\caption{Units for Magnetic Properties}
\label{table}
\setlength{\tabcolsep}{3pt}
\begin{tabular}{|p{25pt}|p{75pt}|p{115pt}|}
\hline
Symbol& 
Quantity& 
Conversion from Gaussian and \par CGS EMU to SI$^{\mathrm{a}}$ \\
\hline
$\Phi $& 
Magnetic flux& 
1 Mx $\to  10^{-8}$ Wb $= 10^{-8}$ V $\cdot$ s \\
$B$& 
Magnetic flux density, \par magnetic induction& 
1 G $\to  10^{-4}$ T $= 10^{-4}$ Wb/m$^{2}$ \\
$H$& 
Magnetic field strength& 
1 Oe $\to  10^{-3}/(4\pi )$ A/m \\
$m$& 
Magnetic moment& 
1 erg/G $=$ 1 emu \par $\to 10^{-3}$ A $\cdot$ m$^{2} = 10^{-3}$ J/T \\
$M$& 
Magnetization& 
1 erg/(G $\cdot$ cm$^{3}) =$ 1 emu/cm$^{3}$ \par $\to 10^{-3}$ A/m \\
4$\pi M$& 
Magnetization& 
1 G $\to  10^{-3}/(4\pi )$ A/m \\
$\sigma $& 
Specific magnetization& 
1 erg/(G $\cdot$ g) $=$ 1 emu/g $\to $ 1 A $\cdot$ m$^{2}$/kg \\
$j$& 
Magnetic dipole \par moment& 
1 erg/G $=$ 1 emu \par $\to 4\pi \times  10^{-10}$ Wb $\cdot$ m \\
$J$& 
Magnetic polarization& 
1 erg/(G $\cdot$ cm$^{3}) =$ 1 emu/cm$^{3}$ \par $\to 4\pi \times  10^{-4}$ T \\
$\chi , \kappa $& 
Susceptibility& 
1 $\to  4\pi $ \\
$\chi_{\rho }$& 
Mass susceptibility& 
1 cm$^{3}$/g $\to  4\pi \times  10^{-3}$ m$^{3}$/kg \\
$\mu $& 
Permeability& 
1 $\to  4\pi \times  10^{-7}$ H/m \par $= 4\pi \times  10^{-7}$ Wb/(A $\cdot$ m) \\
$\mu_{r}$& 
Relative permeability& 
$\mu \to \mu_{r}$ \\
$w, W$& 
Energy density& 
1 erg/cm$^{3} \to  10^{-1}$ J/m$^{3}$ \\
$N, D$& 
Demagnetizing factor& 
1 $\to  1/(4\pi )$ \\
\hline
\multicolumn{3}{p{251pt}}{Vertical lines are optional in tables. Statements that serve as captions for 
the entire table do not need footnote letters. }\\
\multicolumn{3}{p{251pt}}{$^{\mathrm{a}}$Gaussian units are the same as cg emu for magnetostatics; Mx 
$=$ maxwell, G $=$ gauss, Oe $=$ oersted; Wb $=$ weber, V $=$ volt, s $=$ 
second, T $=$ tesla, m $=$ meter, A $=$ ampere, J $=$ joule, kg $=$ 
kilogram, H $=$ henry.}
\end{tabular}
\label{tab1}
\end{table}

\section{Guidelines for Graphics Preparation and Submission}

\subsection{Types of Graphics}

\begin{enumerate}
\item[1)]{\it Color/Grayscale figures:} Figures that are meant to appear in color, or shades of black/gray. Such figures may include photographs, 
         illustrations, multicolor graphs, and flowcharts.
\item[2)]{\it Line Art figures:} Figures that are composed of only black lines and shapes. These figures should have no shades or half-tones of gray, only black and white.
\item[3)]{\it Tables:} Data charts are typically black and white but sometimes include color.
\end{enumerate}

\subsection{Multipart Figures}
Figures compiled of more than one sub-figure presented side-by-side, or stacked. If a multipart figure is made up of multiple figure types (one part is lineart, and another is grayscale or color) the figure should meet the stricter guidelines.

\subsection{File Formats for Graphics}

Format and save your graphics using a suitable graphics processing program that will allow you to create the images as PostScript (PS), Encapsulated PostScript (.EPS), Tagged Image File Format (.TIFF), Portable Document Format (.PDF), or Portable Network Graphics (.PNG); sizes them; and adjusts the resolution settings. If you created your source files in one of the following programs you will be able to submit the graphics without converting to a PS, EPS, TIFF, PDF, or PNG file: Microsoft Word, Microsoft PowerPoint, or Microsoft Excel. Though it is not required, it is strongly recommended that these files be saved in PDF format rather than DOC, XLS, or PPT. Doing so will protect your figures from common font and arrow stroke issues that occur when working on the files across multiple platforms. When submitting your final paper, your graphics should all be submitted individually in one of these formats along with the manuscript.

\subsection{Sizing of Graphics}

Most charts, graphs, and tables are one column wide (3.5 inches / 88 millimeters / 21 picas) or page wide (7.16 inches / 181 millimeters / 43 picas). The maximum height a graphic can be is 8.5 inches (216 millimeters / 54 picas). When choosing the height of a graphic, please allow space for a caption. Figures can be sized between column and page widths if the author chooses, however it is recommended that figures are not sized less than column width unless absolutely necessary. 

\subsection{Resolution}

The proper resolution of your figures will depend on the type of figure it is as defined in the ``Types of Figures'' section. Author photographs, color, and grayscale figures should be at least 300 dpi. Line art, including tables should be a minimum of 600 dpi.


\subsection{Vector Art}

In order to preserve the figures' integrity across multiple computer platforms, we accept files in the following formats: .EPS/.PDF/.PS. All fonts must be embedded or text converted to outlines in order to achieve the best-quality results.

\subsection{Accepted Fonts Within Figures}

When preparing your graphics, IEEE suggests that you use of one of the following Open Type fonts: Times New Roman, Helvetica, Arial, Cambria, and Symbol. If you are supplying EPS, PS, or PDF files all fonts must be embedded. Some fonts may only be native to your operating system; without the fonts embedded, parts of the graphic may be distorted or missing.

A safe option when finalizing your figures is to strip out the fonts before you save the files, creating ``outline'' type. This converts fonts to artwork what will appear uniformly on any screen.


\subsection{Using Labels Within Figures}

\subsubsection{Figure Axis labels}

Figure axis labels are often a source of confusion. Use words rather than symbols. As an example, write the quantity ``Magnetization,'' or ``Magnetization {\em M},'' not just ``{\em M}.'' Put units in parentheses. Do not label axes only with units. As in Fig. 1, for example, write ``Magnetization (A/m)'' or ``Magnetization (A$\cdot$m$^{-1}$),'' not just ``A/m.'' Do not label axes with a ratio of quantities and units. For example, write ``Temperature (K),'' not ``Temperature/K.''

Multipliers can be especially confusing. Write ``Magnetization (kA/m)'' or ``Magnetization (10$^{3}$ A/m).'' Do not write ``Magnetization (A/m) $\times$ 1000'' because the reader would not know whether the top axis label in Fig. 1 meant 16000 A/m or 0.016 A/m. Figure labels should be legible, approximately 8 to 10 point type. 

\subsubsection{Subfigure Labels in Multipart Figures and Tables}

Multipart figures should be combined and labeled before final submission. Labels should appear centered below each subfigure in 8 point Times New Roman font in the format of (a), (b), (c). 

\subsection{File Naming}

Figures (line artwork or photographs) should be named starting with the first 5 letters of the author’s last name. The next characters in the filename should be the number that represents the sequential location of this image in your article. For example, in author ``Anderson’s'' paper, the first three figures would be named ander1.tif, ander2.tif, and ander3.ps.

Tables should contain only the body of the table (not the caption) and should be named similarly to figures, except that `.t' is inserted in-between the author’s name and the table number. For example, author Anderson’s first three tables would be named ander.t1.tif, ander.t2.ps, ander.t3.eps.


\subsection{Referencing a Figure or Table Within Your Paper}

When referencing your figures and tables within your paper, use the abbreviation ``Fig.'' even at the beginning of a sentence. Do not abbreviate ``Table.'' Tables should be numbered with Roman Numerals.


\subsection{Checking Your Figures: The IEEE Graphics Analyzer}


The IEEE Graphics Analyzer enables authors to pre-screen their graphics for compliance with IEEE Transactions and Journals standards before submission. The online tool, located at http://graphicsqc.ieee.org/, allows authors to upload their graphics in order to check that each file is the correct file format, resolution, size and colorspace; that no fonts are missing or corrupt; that figures are not compiled in layers or have transparency, and that they are named according to the IEEE Transactions and Journals naming convention. At the end of this automated process, authors are provided with a detailed report on each graphic within the web applet, as well as by email.

For more information on using the Graphics Analyzer or any other graphics related topic, contact the IEEE Graphics Help Desk by e-mail at graphics@ieee.org.

\subsection{Submitting Your Graphics}
Because IEEE will do the final formatting of your paper, 
you do not need to position figures and tables at the top and bottom of each column. In fact, all figures, figure captions, and tables can be placed at the end of your paper. In addition to, or even in lieu of submitting figures within your final manuscript, figures should be submitted individually, separate from the manuscript in one of the file formats listed above in Section VI-J. Place figure captions below the figures; place table titles above the tables. Please do not include captions as part of the figures, or put them in ``text boxes'' linked to the figures. Also, do not place borders around the outside of your figures.


\section{Conclusion}

A conclusion section is not required. Although a conclusion may review the main points of the paper, do not replicate the abstract as the conclusion. A conclusion might elaborate on the importance of the work or suggest applications and extensions. 

\section{Acknowledgment}

The preferred spelling of the word ``acknowledgment'' in American English is without an ``e'' after the ``g.'' Use the singular heading even if you have many acknowledgments. Avoid expressions such as ``One of us (S.B.A.) would like to thank . . . .'' Instead, write “F. A. Author thanks ... .” In most cases, sponsor and financial support acknowledgments are placed in the unnumbered footnote on the first page, not here.

\section{References and Footnotes}

\subsection{References}

References need not be cited in text. When they are, they appear on the line, in square brackets, inside the punctuation.  Multiple references are each numbered with separate brackets. When citing a section in a book, please give the relevant page numbers. In text, refer simply to the reference number. Do not use ``Ref.'' or ``reference'' except at the beginning of a sentence: ``Reference [3] shows . . . .'' Please do not use automatic endnotes in {\em Word}, rather, type the reference list at the end of the paper using the ``References'' style.

Reference numbers are set flush left and form a column of their own, hanging out beyond the body of the reference. The reference numbers are on the line, enclosed in square brackets. In all references, the given name of the author or editor is abbreviated to the initial only and precedes the last name. Use them all; use {\em et al.} only if names are not given. Use commas around Jr., Sr., and III in names. Abbreviate conference titles.  When citing IEEE transactions, provide the issue number, page range, volume number, year, and/or month if available. When referencing a patent, provide the day and the month of issue, or application. References may not include all information; please obtain and include relevant information. Do not combine references. There must be only one reference with each number. If there is a URL included with the print reference, it can be included at the end of the reference.

Other than books, capitalize only the first word in a paper title, except for proper nouns and element symbols. For papers published in translation journals, please give the English citation first, followed by the original foreign-language citation. See the end of this document for formats and examples of common references. For a complete discussion of references and their formats, see the IEEE style manual at www.ieee.org/authortools.

\subsection{Footnotes}

Number footnotes separately in superscripts (Insert $\mid$ Footnote).\footnote{It is recommended that footnotes be avoided (except for the unnumbered footnote with the receipt date on the first page). Instead, try to integrate the footnote information into the text.}  Place the actual footnote at the bottom of the column in which it is cited; do not put footnotes in the reference list (endnotes). Use letters for table footnotes (see Table I). 


\section*{References}

\subsection*{Basic format for books:}

J. K. Author, ``Title of chapter in the book,'' in {\em Title of His Published Book}, xth ed. City of Publisher, (only U.S. State), Country: Abbrev. of Publisher, year, ch. x, sec. x, pp. xxx--xxx.

\subsection*{Examples:}

\begin{thebibliography}{34}

\bibitem{}G. O. Young, ``Synthetic structure of industrial plastics,'' in {\em Plastics}, 2nd ed., vol. 3, J. Peters, Ed. New York, NY, USA: McGraw-Hill, 1964, pp. 15--64.

\bibitem{}W.-K. Chen, {\it Linear Networks and Systems}. Belmont, CA, USA: Wadsworth, 1993, pp. 123--135.

\end{thebibliography}

\subsection*{Basic format for periodicals:}

J. K. Author, ``Name of paper,'' Abbrev. Title of Periodical, vol. x,   no. x, pp. xxx--xxx, Abbrev. Month, year, DOI. 10.1109.XXX.123--456.

\subsection*{Examples:}

\begin{thebibliography}{34}
\setcounter{enumiv}{2}

\bibitem{}J. U. Duncombe, ``Infrared navigation Part I: An assessment of feasibility,'' {\em IEEE Trans. Electron Devices}, vol. ED-11, no. 1, pp. 34--39, Jan. 1959,10.1109/TED.2016.2628402.

\bibitem{}E. P. Wigner, ``Theory of traveling-wave optical laser,''
{\em Phys. Rev.},  vol. 134, pp. A635--A646, Dec. 1965.

\bibitem{}E. H. Miller, ``A note on reflector arrays,'' {\em IEEE Trans. Antennas Propagat.}, to be published.
\end{thebibliography}


\subsection*{Basic format for reports:}

J. K. Author, ``Title of report,'' Abbrev. Name of Co., City of Co., Abbrev. State, Country, Rep. xxx, year.

\subsection*{Examples:}
\begin{thebibliography}{34}
\setcounter{enumiv}{5}

\bibitem{} E. E. Reber, R. L. Michell, and C. J. Carter, ``Oxygen absorption in the earth’s atmosphere,'' Aerospace Corp., Los Angeles, CA, USA, Tech. Rep. TR-0200 (4230-46)-3, Nov. 1988.

\bibitem{} J. H. Davis and J. R. Cogdell, ``Calibration program for the 16-foot antenna,'' Elect. Eng. Res. Lab., Univ. Texas, Austin, TX, USA, Tech. Memo. NGL-006-69-3, Nov. 15, 1987.
\end{thebibliography}

\subsection*{Basic format for handbooks:}

{\em Name of Manual/Handbook}, x ed., Abbrev. Name of Co., City of Co., Abbrev. State, Country, year, pp. xxx--xxx.

\subsection*{Examples:}

\begin{thebibliography}{34}
\setcounter{enumiv}{7}

\bibitem{} {\em Transmission Systems for Communications}, 3rd ed., Western Electric Co., Winston-Salem, NC, USA, 1985, pp. 44--60.

\bibitem{} {\em Motorola Semiconductor Data Manual}, Motorola Semiconductor Products Inc., Phoenix, AZ, USA, 1989.
\end{thebibliography}

\subsection*{Basic format for books (when available online):}

J. K. Author, ``Title of chapter in the book,'' in {\em Title of Published Book}, xth ed. City of Publisher, State, Country: Abbrev. of Publisher, year, ch. x, sec. x, pp. xxx xxx. [Online]. Available: http://www.web.com 

\subsection*{Examples:}

\begin{thebibliography}{34}
\setcounter{enumiv}{9}

\bibitem{}G. O. Young, ``Synthetic structure of industrial plastics,'' in Plastics, vol. 3, Polymers of Hexadromicon, J. Peters, Ed., 2nd ed. New York, NY, USA: McGraw-Hill, 1964, pp. 15--64. [Online]. Available: http://www.bookref.com. 

\bibitem{} {\em The Founders Constitution}, Philip B. Kurland and Ralph Lerner, eds., Chicago, IL, USA: Univ. Chicago Press, 1987. [Online]. Available: http://press-pubs.uchicago.edu/founders/

\bibitem{} The Terahertz Wave eBook. ZOmega Terahertz Corp., 2014. [Online]. Available: http://dl.z-thz.com/eBook/zomega\_ebook\_pdf\_1206\_sr.pdf. Accessed on: May 19, 2014. 

\bibitem{} Philip B. Kurland and Ralph Lerner, eds., {\em The Founders Constitution}. Chicago, IL, USA: Univ. of Chicago Press, 1987, Accessed on: Feb. 28, 2010, [Online] Available: http://press-pubs.uchicago.edu/founders/ 
\end{thebibliography}

\subsection*{Basic format for journals (when available online):}

J. K. Author, ``Name of paper,'' {\em Abbrev. Title of Periodical}, vol. x, no. x, pp. xxx--xxx, Abbrev. Month, year. Accessed on: Month, Day, year, doi: 10.1109.XXX.123456, [Online].

\subsection*{Examples:}

\begin{thebibliography}{34}
\setcounter{enumiv}{13}

\bibitem{}J. S. Turner, ``New directions in communications,'' {\em IEEE J. Sel. Areas Commun.}, vol. 13, no. 1, pp. 11--23, Jan. 1995. 

\bibitem{} W. P. Risk, G. S. Kino, and H. J. Shaw, ``Fiber-optic frequency shifter using a surface acoustic wave incident at an oblique angle,'' {\em Opt. Lett.}, vol. 11, no. 2, pp. 115--117, Feb. 1986.

\bibitem{} P. Kopyt {\em et al.}, ``Electric properties of graphene-based conductive layers from DC up to terahertz range,'' {\em IEEE THz Sci. Technol.}, to be published. doi: 10.1109/TTHZ.2016.2544142.
\end{thebibliography}

\subsection*{Basic format for papers presented at conferences (when available online):}

J.K. Author. (year, month). Title. presented at abbrev. conference title. [Type of Medium]. Available: site/path/file

\subsection*{Example:}

\begin{thebibliography}{34}
\setcounter{enumiv}{16}

\bibitem{}PROCESS Corporation, Boston, MA, USA. Intranets: Internet technologies deployed behind the firewall for corporate productivity. Presented at INET96 Annual Meeting. [Online]. Available: http://home.process.com/Intranets/wp2.htp
\end{thebibliography}

\subsection*{Basic format for reports  and  handbooks (when available online):}
  
J. K. Author. ``Title of report,'' Company. City, State, Country. Rep. no., (optional: vol./issue), Date. [Online] Available: site/path/file 

\subsection*{Examples:}

\begin{thebibliography}{34}
\setcounter{enumiv}{17}

\bibitem{}R. J. Hijmans and J. van Etten, ``Raster: Geographic analysis and modeling with raster data,'' R Package Version 2.0-12, Jan. 12, 2012. [Online]. Available: http://CRAN.R-project.org/package=raster 

\bibitem{}Teralyzer. Lytera UG, Kirchhain, Germany [Online]. Available: http://www.lytera.de/Terahertz\_THz\_Spectroscopy.php?id=home, Accessed on: Jun. 5, 2014.
\end{thebibliography}

\subsection*{Basic format for computer programs and electronic documents (when available online):}

Legislative body. Number of Congress, Session. (year, month day). {\em Number of bill or resolution, Title}. [Type of medium]. Available: site/path/file
{\em NOTE:} ISO recommends that capitalization follow the accepted practice for the language or script in which the information is given.

\subsection*{Example:}

\begin{thebibliography}{34}
\setcounter{enumiv}{19}

\bibitem{}U. S. House. 102nd Congress, 1st Session. (1991, Jan. 11). {\em H. Con. Res. 1, Sense of the Congress on Approval of Military Action}. [Online]. Available: LEXIS Library: GENFED File: BILLS 
\end{thebibliography}

\subsection*{Basic format for patents (when available online):}

Name of the invention, by inventor’s name. (year, month day). Patent Number [Type of medium]. Available:site/path/file

\subsection*{Example:}

\begin{thebibliography}{34}
\setcounter{enumiv}{20}

\bibitem{}Musical tooth brush with mirror, by L. M. R. Brooks. (1992, May 19). Patent D 326 189
[Online]. Available: NEXIS Library: LEXPAT File:   DES 

\end{thebibliography}

\subsection*{Basic format for conference proceedings (published):}

J. K. Author, ``Title of paper,'' in {\em Abbreviated Name of Conf.}, City of Conf., Abbrev. State (if given), Country, year, pp. xxx--xxx.

\subsection*{Example:}

\begin{thebibliography}{34}
\setcounter{enumiv}{21}

\bibitem{}D. B. Payne and J. R. Stern, ``Wavelength-switched passively coupled single-mode optical network,'' in {\em Proc. IOOC-ECOC}, Boston, MA, USA, 1985,
pp. 585--590.

\end{thebibliography}

\subsection*{Example for papers presented at conferences (unpublished):}

\begin{thebibliography}{34}
\setcounter{enumiv}{22}

\bibitem{}D. E behard and E. Voges, ``Digital single sideband detection for inter ferometric sensors,'' presented at the {\em 2nd Int. Conf. Optical Fiber Sensors}, Stuttgart, Germany, Jan. 2--5, 1984.
\end{thebibliography}

\subsection*{Basic formatfor patents:}

J. K. Author, ``Title of patent,'' U. S. Patent x xxx xxx, Abbrev. Month, day, year.

\subsection*{Example:}

\begin{thebibliography}{34}
\setcounter{enumiv}{23}

\bibitem{}G. Brandli and M. Dick, ``Alternating current fed power supply,'' U. S. Patent 4 084 217, Nov. 4, 1978.
\end{thebibliography}

\subsection*{Basic format for theses (M.S.) and dissertations (Ph.D.):}

a) J. K. Author, ``Title of thesis,'' M. S. thesis, Abbrev. Dept., Abbrev. Univ., City of Univ., Abbrev. State, year.

b) J. K. Author, ``Title of dissertation,'' Ph.D. dissertation, Abbrev. Dept., Abbrev. Univ., City of Univ., Abbrev. State, year.

\subsection*{Examples:}

\begin{thebibliography}{34}
\setcounter{enumiv}{24}

\bibitem{}J. O. Williams, ``Narrow-band analyzer,'' Ph.D. dissertation, Dept. Elect. Eng., Harvard Univ., Cambridge, MA, USA, 1993.

\bibitem{}N. Kawasaki, ``Parametric study of thermal and chemical nonequilibrium nozzle flow,'' M.S. thesis, Dept. Electron. Eng., Osaka Univ., Osaka, Japan, 1993.
\end{thebibliography}

\subsection*{Basic format for the most common types of unpublished references:}

a) J. K. Author, private communication, Abbrev. Month, year.

b) J. K. Author, ``Title of paper,'' unpublished.

c) J. K. Author, ``Title of paper,'' to be published.

\subsection*{Examples:}

\begin{thebibliography}{34}
\setcounter{enumiv}{26}

\bibitem{}A. Harrison, private communication, May 1995.

\bibitem{}B. Smith, ``An approach to graphs of linear forms,'' unpublished.

\bibitem{}A. Brahms, ``Representation error for real numbers in binary computer arithmetic,'' IEEE Computer Group Repository, Paper R-67-85.
\end{thebibliography}

\subsection*{Basic formats for standards:}

a) {\em Title of Standard}, Standard number, date.

b) {\em Title of Standard}, Standard number, Corporate author, location, date.

\subsection*{Examples:}

\begin{thebibliography}{34}
\setcounter{enumiv}{29}


\bibitem{}IEEE Criteria for Class IE Electric Systems, IEEE Standard 308, 1969.

\bibitem{} Letter Symbols for Quantities, ANSI Standard Y10.5-1968.
\end{thebibliography}

\subsection*{Article number in reference examples:}

\begin{thebibliography}{34}
\setcounter{enumiv}{31}

\bibitem{}R. Fardel, M. Nagel, F. Nuesch, T. Lippert, and A. Wokaun, ``Fabrication of organic light emitting diode pixels by laser-assisted forward transfer,'' {\em Appl. Phys. Lett.}, vol. 91, no. 6, Aug. 2007, Art. no. 061103. 

\bibitem{} J. Zhang and N. Tansu, ``Optical gain and laser characteristics of InGaN quantum wells on ternary InGaN substrates,'' {\em IEEE Photon.} J., vol. 5, no. 2, Apr. 2013, Art. no. 2600111
\end{thebibliography}

\subsection*{Example when using et al.:}

\begin{thebibliography}{34}
\setcounter{enumiv}{33}

\bibitem{}S. Azodolmolky {\em et al.}, Experimental demonstration of an impairment aware network planning and operation tool for transparent/translucent optical networks,'' {\em J. Lightw. Technol.}, vol. 29, no. 4, pp. 439--448, Sep. 2011.
\end{thebibliography}

\end{document}